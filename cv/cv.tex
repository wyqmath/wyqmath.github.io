\documentclass[11pt]{article}
\usepackage{graphicx}
\usepackage{helvet}
\renewcommand{\familydefault}{\sfdefault}

\setlength{\parindent}{0pt}
\usepackage{hyperref}
\usepackage{enumitem}
\usepackage[utf8]{inputenc} 
\usepackage[T1]{fontenc}
\usepackage[english]{babel}
\usepackage[left=1.06cm,top=1.7cm,right=1.06cm,bottom=0.49cm]{geometry}

\usepackage{multicol}
\usepackage{xcolor}
\definecolor{mylinkcolor}{RGB}{0,0,150}
\hypersetup{
	colorlinks=true,
	linkcolor=mylinkcolor,
	filecolor=mylinkcolor,      
	urlcolor=mylinkcolor,
	citecolor=mylinkcolor,
}

\begin{document}
	\begin{center}
		\textbf{\Large Yiquan Wang}\\
		\textbullet \ +86-19537838515 \textbullet \ \href{mailto:ethan@stu.xju.edu.cn}{ethan@stu.xju.edu.cn}\ \textbullet \ \href{https://wyqmath.cn/en.html}{Personal Homepage} \\
		\textbullet \ \href{https://github.com/wyqmath}{GitHub} \textbullet \ \href{https://scholar.google.com/citations?user=ULP3e1cAAAAJ}{Google Scholar} \textbullet \ \href{https://orcid.org/0000-0003-1417-5752}{ORCID} \textbullet \ \href{https://openreview.net/profile?id=~Yiquan_Wang3}{OpenReview} \textbullet \ IEEE Biometrics Council Member \\
		\hrulefill
	\end{center}
	
	\vspace{2pt}
	
	\begin{center}
		\textbf{\large Education}
	\end{center}
	\textbf{National Base for Mathematical Research and Teaching Talents, Xinjiang University} \hfill Urumqi, Xinjiang
	
	\textit{B.S. in Mathematics and Applied Mathematics} \hfill 2023.09 – 2027.06
	
	\vspace{10pt}
	
	\textbf{Tsien Excellence in Engineering Program, Tsinghua University \& X-Institute} \hfill Shenzhen, Guangdong
	
	\textit{Joint Program, Zero One Scholar} \hfill 2024.06 – 2027.06
	
	\vspace{10pt}
	
	\textbf{Institute of Neurological and Psychiatric Disorders, Shenzhen Bay Laboratory} \hfill Shenzhen, Guangdong
	
	\textit{Visiting Student, Wen Yuan's Research Group} \hfill 2025.07 – 2025.09
	
	\vspace{12pt}
	
	\begin{center}
		\textbf{\large Research Projects}
	\end{center}
	\textbf{National Undergraduate Innovation Training Program} \hfill 2025.04 – 2026.04
	\begin{itemize}[noitemsep, topsep=0pt, partopsep=0pt, parsep=0pt, leftmargin=*]
		\item \textbf{Project:} Copy Number Variation Conditional Diffusion Model: For Alzheimer's Disease Risk Assessment.
		\item This project aims to integrate copy number variation (CNV) features from whole-genome sequencing data with multi-dimensional clinical data such as metabolic indicators. Leveraging the success of diffusion models in processing high-dimensional simulation data and protein phenotype prediction, we will construct a comprehensive framework consisting of CNV feature encoding, genomic region attention, and conditional U-Net diffusion modules. This will simulate CNV distribution changes and evolutionary processes in the genome, analyze the specific role of CNV in regulating Alzheimer's disease pathways, and ultimately improve disease risk assessment and early intervention accuracy. Supervised by Prof. Kai Wei.
	\end{itemize}
	
	\vspace{10pt}
	
	\textbf{Chinese Academy of Sciences (CAS) Innovation Practice Training Program} \hfill 2024.11 – 2025.09
	\begin{itemize}[noitemsep, topsep=0pt, partopsep=0pt, parsep=0pt, leftmargin=*]
		\item \textbf{Project:} Extraction and Analysis of Global Heatwave Disaster Adaptation Elements Based on Multimodal BERT Model.
		\item This research uses a multimodal BERT model to integrate text, images, and structured data to precisely identify key factors affecting heatwave adaptation. The research results will provide solid theoretical and data support for formulating scientific global heatwave response strategies and enhance society's overall disaster adaptation capabilities. Supervised by Researcher Yong Ge.
	\end{itemize}
	
	\vspace{10pt}
	
	\textbf{Tsinghua University Tsien Excellence in Engineering Program ESRT} \hfill 2024.08-2025.10
	\begin{itemize}[noitemsep, topsep=0pt, partopsep=0pt, parsep=0pt, leftmargin=*]
		\item \textbf{Project:} From Signal to Symphony: Predicting Protein Function with a Deep Learning Fusion Model on Sonified Sequences.
		\item This research introduces a novel computational framework that translates protein sequence and structural information into musical encodings to predict function and guide protein design. The study systematically demonstrates the value of co-evolving both data representation and model complexity through a three-stage comparative analysis, achieving 90.44\% accuracy. The framework was also integrated into a conditional diffusion model to guide the generative design of novel, structurally viable Green Fluorescent Protein (GFP) variants. Supervised by Prof. Kai Wei.
	\end{itemize}
	
	\vspace{10pt}
	
	\textbf{Provincial Undergraduate Innovation Training Program} \hfill 2024.03 – 2025.06
	\begin{itemize}[noitemsep, topsep=0pt, partopsep=0pt, parsep=0pt, leftmargin=*]
		\item \textbf{Project:} Research on the Generation Cyclability of Cartesian Product Graphs and Related Problems.
		\item This project mainly focuses on studying the generation cyclability of the n-th Cartesian product graphs of complete graphs. Currently, there are research results on the generation cyclability of n-th Cartesian product graphs of complete graphs with two vertices and three vertices. I mainly study the generation cyclability of n-th Cartesian product graphs of complete graphs with at least 4 vertices. Supervised by Prof. Eminjan Sabir.
	\end{itemize}
	
	\vspace{12pt}
	
	\clearpage
	\begin{center}
		\textbf{\large Professional Experience}
	\end{center}
	\textbf{Beijing Frontier Research Center for Biological Structure, Tsinghua University} \hfill Beijing, China
	
	\textit{Research Intern} \hfill 2026.07 – Present
	\begin{itemize}[noitemsep, topsep=0pt, partopsep=0pt, parsep=0pt, leftmargin=*]
		\item Developed the \href{https://www.frcbs.tsinghua.edu.cn/cpdb/}{Polypeptide Structure Database} and engaged in protein and polypeptide design research.
	\end{itemize}
	
	\vspace{10pt}
	
	\textbf{Institute of Software, CAS \& Huawei Mindspore Community} \hfill Remote
	
	\textit{Open Source Intern} \hfill 2024.09 – 2025.03
	\begin{itemize}[noitemsep, topsep=0pt, partopsep=0pt, parsep=0pt, leftmargin=*]
		\item Implemented a VGG19-based model for Pollock-style art generation, focusing on fractal and turbulent feature extraction and the creation of NFT-based digital labels. Featured on \href{https://mp.weixin.qq.com/s/_N5oLsa0etJTxEOKRTt_4w}{Huawei's official Wechat}.
	\end{itemize}
	
	\vspace{10pt}
	
	\vspace{12pt}
	
	\begin{center}
		\textbf{\large Selected Publications}
	\end{center}
	\begin{enumerate}[noitemsep, topsep=0pt, partopsep=0pt, parsep=0pt, leftmargin=*, label={[\arabic*]}]
		\item \textbf{Wang, Y.*}, Cai, M., \& Huang, T. Y. (2025). AI for disease prediction: Performance insights and key limitations. \textit{Journal of Clinical Neuroscience, 138}, 111360.
		\item Wang, X., \textbf{Wang, Y.}, \& Huang, T. Y. (2025). Crypto-ncRNA: Non-coding RNA (ncRNA) Based Encryption Algorithm. \textit{ICLR 2025 Workshop}. (Co-first author).
		\item Wang, X., \textbf{Wang, Y.*}, \& Pan, J. (2025). Digital Art Creation and Copyright Protection in Pollock Style Using GANs, Fractal Analysis, and NFT Generation. \textit{ICLR 2025 Workshop}. (Co-first author).
		\item Wang, X., Xu, L., \textbf{Wang, Y.*}, Dong, Y., Li, X., Deng, J., \& He, R. (2024). Octopus Inspired Optimization Algorithm: Multi-Level Structures and Parallel Computing Strategies. \textit{arXiv preprint arXiv:2410.07968}.
		\item Wang, X., Wang, F., \& \textbf{Wang, Y.} (2025). Dialogues between adam and eve: exploration of unknown civilization language by llm. \textit{ICLR 2025 Workshop}.
		\item \textbf{Wang, Y.*}, Zhang, J., \& Chang, Y. (2024, November). A probability prediction model for flood disasters based on Multi-layer Perceptron. In \textit{Journal of Physics: Conference Series} (Vol. 2905, No. 1, p. 012003). IOP Publishing.
		\item Wang, J., \& \textbf{Wang, Y.*} (2024, September). Multi-stage Crop Planting Strategy optimization Model Based on PSO Algorithm. In \textit{2024 3rd International Conference on Electronics and Information Technology (EIT)} (pp. 915-919). IEEE.
		\item \textbf{Wang, Y.}, Cai, M., Dong, Y., et al. (2025). From Signal to Symphony: Predicting Protein Function with a Deep Learning Fusion Model on Sonified Sequences (under review).
		\item \textbf{Wang, Y.}, Zhang, J., Chang, Y., et al. (2025). Boosting MOOCs Engagement through Graph Neural Network-Driven Social-Academic Recommendations. (under review).
		\item \textbf{Wang, Y.*}, Wang, J., Huang, T. Y., et al. (2025). STGCN-LSTM for Olympic Medal Prediction: Dynamic Power Modeling and Causal Policy Optimization.  (under review).
	\end{enumerate}
	
	\vspace{12pt}
	
	\begin{center}
		\textbf{\large Awards \& Honors}
	\end{center}
	\begin{itemize}[noitemsep, topsep=0pt, partopsep=0pt, parsep=0pt, leftmargin=*]
		\item \textbf{SynBio Challenges}: Silver Award \hfill 2025
		\item \textbf{The Mathematical Contest in Modeling (MCM)}: Honorable Mention \hfill 2025
		\item \textbf{Alibaba Cloud Tianchi University Student Competition}: National Finals, 17th Place \hfill 2024
		\item \textbf{APMCM Asia-Pacific Mathematical Modeling Competition}: National Third Prize \hfill 2024
		\item \textbf{“Alpha Egg Cup” National Go Championship}: 15th Place \hfill 2024
		\item \textbf{Xinjiang Youth Amateur Go Competition}: 53rd Place \hfill 2024
		\item \textbf{Hunan Province Spring Cup Go Competition}: 7th Place \hfill 2024
		\item \textbf{National Youth Intellectual Sports Meeting Go Competition}: 9th Place \hfill 2024
		\item \textbf{Xinjiang University Vulnerability Reporting Honor} \hfill 2023
		\item \textbf{“Tianshan Fixed Network Cup” Cybersecurity Skills Competition}: 7th Place, Xinjiang Region \hfill 2023
	\end{itemize}

	\clearpage
	\begin{center}
		\textbf{\large Academic Activities}
	\end{center}
	\textbf{Peer Reviewer:}
	\begin{multicols}{2}
		\begin{itemize}[noitemsep, topsep=0pt, partopsep=0pt, parsep=0pt, leftmargin=*]
			\item NeurIPS 2025 AI for Science Workshop
			\item ICLR 2025 Workshop on AI for Nucleic Acids
			\item ICLR 2025 Workshop on GenAI Watermarking
			\item ICML 2025 Workshop on AI for Math
			\item Mini-Reviews in Medicinal Chemistry
			\item Current Science
			\item F1000 Research
		\end{itemize}
	\end{multicols}
	
	\textbf{Academic Engagement:}
	\begin{itemize}[noitemsep, topsep=0pt, partopsep=0pt, parsep=0pt, leftmargin=*]
		\item Tsinghua University-Peking University Center for Life Sciences Summer Camp \hfill 2025.07
		\item \begin{tabular}[c]{@{}p{0.8\linewidth}@{}}Shenzhen Bay Laboratory / Shenzhen Medical Academy of Research and Translation Summer Research\end{tabular} \hfill \mbox{2025.07 -- 2025.09}
		\item Tsinghua University Tsien Excellence in Engineering Program -- Zero One Scholar \hfill 2024.06 -- 2027.06
		\item AI Winter School, Brown University Department of Physics \hfill 2025.01
		\item CAAI Artificial Intelligence and Technology Ethics Training Course \hfill 2024.09 -- 2024.12
		\item Fudan University Summer School of Mathematical Logic \hfill 2024.08
		\item Jinan University Guangdong Thousand Villages Survey Project \hfill 2024.08
		\item Wuhan University National Tianyuan Mathematics Center Discussion Class \hfill 2024.03 -- 2024.06
	\end{itemize}
	
	\vspace{12pt}

	\begin{center}
		\textbf{\large Website Development}
	\end{center}
	\begin{itemize}[noitemsep, topsep=0pt, partopsep=0pt, parsep=0pt, leftmargin=*]
		\item \textbf{Shenzhen X-Institute Course Website}: \href{https://lingyi.wyqmath.cn/}{https://lingyi.wyqmath.cn/}
		\item \textbf{Polypeptide Structure Database}: \href{https://www.frcbs.tsinghua.edu.cn/cpdb/}{https://www.frcbs.tsinghua.edu.cn/cpdb/}
		\item \textbf{Tong Wang Research Group}: \href{https://tongwang.vercel.app/}{https://tongwang.vercel.app/}
	\end{itemize}

	\vspace{12pt}

	\begin{center}
		\textbf{\large Skills \& Interests}
	\end{center}
	\textbf{Technical Skills:} Python, C/C++, MATLAB, LaTeX, Linux, HTML/CSS/JavaScript
	
	\textbf{Research Interests:} Artificial Intelligence, Deep Learning, AI for Science, Bioinformatics, Computational Biology, Mathematical Modeling, Neuroscience
	
	\textbf{Languages:} Chinese (Native), English (Professional Working Proficiency)
	
\end{document}