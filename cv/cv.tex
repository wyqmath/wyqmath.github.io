\documentclass[11pt]{article}
\usepackage{graphicx}
\usepackage{helvet}
\renewcommand{\familydefault}{\sfdefault}

\setlength{\parindent}{0pt}
\usepackage{hyperref}
\usepackage{enumitem}
\usepackage[utf8]{inputenc} 
\usepackage[T1]{fontenc}
\usepackage[english]{babel}
\usepackage[left=1.06cm,top=1.7cm,right=1.06cm,bottom=0.49cm]{geometry}

\usepackage{multicol}
\usepackage{xcolor}
\definecolor{mylinkcolor}{RGB}{0,0,150}
\hypersetup{
	colorlinks=true,
	linkcolor=mylinkcolor,
	filecolor=mylinkcolor,      
	urlcolor=mylinkcolor,
	citecolor=mylinkcolor,
}

\begin{document}
	\begin{center}
		\textbf{\Large Yiquan Wang}\\
		\textbullet \ +86-19537838515 \textbullet \ \href{mailto:ethan@stu.xju.edu.cn}{ethan@stu.xju.edu.cn}\ \textbullet \ \href{https://wyqmath.cn/}{Personal Homepage} \\
		\textbullet \ \href{https://github.com/wyqmath}{GitHub} \textbullet \ \href{https://scholar.google.com/citations?user=ULP3e1cAAAAJ}{Google Scholar} \textbullet \ \href{https://orcid.org/0000-0003-1417-5752}{ORCID} \textbullet \ \href{https://openreview.net/profile?id=~Yiquan_Wang3}{OpenReview} \textbullet \ IEEE Biometrics Council Member \\
		\hrulefill
	\end{center}
	
	\vspace{2pt}
	
	\begin{center}
		\textbf{\large Education}
	\end{center}
	\textbf{National Base for Mathematical Research and Teaching Talents, Xinjiang University} \hfill Urumqi, Xinjiang
	
	\textit{B.S. in Mathematics and Applied Mathematics} \hfill 2023.09 – 2027.06
	
	\vspace{10pt}
	
	\textbf{Tsien Excellence in Engineering Program, Tsinghua University \& X-Institute} \hfill Shenzhen, Guangdong
	
	\textit{Joint Program, X Scholar} \hfill 2024.06 – 2027.06
	
	\vspace{10pt}
	
	\textbf{Institute of Neurological and Psychiatric Disorders, Shenzhen Bay Laboratory} \hfill Shenzhen, Guangdong
	
	\textit{Visiting Student, Wen Yuan's Research Group} \hfill 2025.07 – 2025.09
	
	\vspace{12pt}
	
	\begin{center}
		\textbf{\large Research Projects}
	\end{center}
	\textbf{Tsinghua University Tsien Excellence in Engineering Program ESRT} \hfill 2024.08-2025.10
	\begin{itemize}[noitemsep, topsep=0pt, partopsep=0pt, parsep=0pt, leftmargin=*]
		\item \textbf{Project:} From Signal to Symphony: Exploring 2D Sequence Representations for Protein Function Prediction.
		\item {\small Predicting protein function from its primary sequence is a fundamental challenge in computational biology. While deep learning has excelled, the optimal representation of sequence data remains an open question. This study explores protein sonification---the conversion of amino acid sequences into 2D spectrograms---as a representation for this task. To facilitate this investigation, we developed a benchmark dataset of 18,000 sequences spanning 12 functionally diverse protein classes. Our systematic evaluation suggests that the structural transformation from a 1D sequence to a 2D spectrogram may be a key contributor to the model's predictive performance. This observation is supported by ablation studies where models using either purely visual or acoustic features from the spectrogram demonstrated effective standalone performance, suggesting that the representation itself is a key source of this capability. For instance, a model using a sonification map without explicit biophysical meaning achieved 81.08\% accuracy, while our biophysically-informed model reached 84.00\%, indicating that such domain knowledge may offer a modest performance benefit. When trained from scratch on our dataset, our fusion model achieved performance comparable to or slightly exceeding that of standard transformer architectures like ESM-2 and ProtBERT, suggesting its potential for data efficiency in this specific context. The model's potential for generalizability was further supported by its performance on the external CARE enzyme classification benchmark, where it achieved 90.44\% accuracy. Finally, as a proof-of-concept, we explore the utility of our encoding to guide a diffusion model in generating novel GFP variants, which were assessed for structural viability using computational methods. Our work provides evidence suggesting that the utility of sonification in this context may stem largely from its representational structure, offering a perspective on feature engineering for biological sequences.}
		\item \href{https://github.com/wyqmath/Symphony_of_Fate}{Project Repository}. Supervised by Prof. \href{https://scholar.google.com/citations?user=LLj__mMAAAAJ}{Kai Wei}.
	\end{itemize}

	\vspace{10pt}

	\textbf{National Undergraduate Innovation Training Program} \hfill 2025.04 – 2026.04
	\begin{itemize}[noitemsep, topsep=0pt, partopsep=0pt, parsep=0pt, leftmargin=*]
		\item \textbf{Project:} Copy Number Variation Conditional Diffusion Model: For Alzheimer's Disease Risk Assessment.
		\item {\small This project aims to integrate copy number variation (CNV) features from whole-genome sequencing data with multi-dimensional clinical data such as metabolic indicators. Leveraging the success of diffusion models in processing high-dimensional simulation data and protein phenotype prediction, we will construct a comprehensive framework consisting of CNV feature encoding, genomic region attention, and conditional U-Net diffusion modules. This will simulate CNV distribution changes and evolutionary processes in the genome, analyze the specific role of CNV in regulating Alzheimer's disease pathways, and ultimately improve disease risk assessment and early intervention accuracy.}
		\item Supervised by Prof. \href{https://scholar.google.com/citations?user=LLj__mMAAAAJ}{Kai Wei}.
	\end{itemize}

	\vspace{10pt}

	\textbf{Chinese Academy of Sciences (CAS) Innovation Practice Training Program} \hfill 2024.11 – 2025.09
	\begin{itemize}[noitemsep, topsep=0pt, partopsep=0pt, parsep=0pt, leftmargin=*]
		\item \textbf{Project:} A Knowledge Graph-based Q\&A System for Heatwave Disaster Adaptation.
		\item {\small To address the urgent global challenge of increasingly severe heatwaves, this project details an automated pipeline for constructing a domain-specific knowledge graph from academic literature. The system leverages a Large Language Model to extract relational triples from unstructured text. A novel workflow then standardizes entities by using SentenceTransformer embeddings and a FAISS index for semantic clustering, followed by LLM-based canonical name generation. The resulting knowledge graph powers a hybrid KG-RAG query engine that retrieves factual subgraphs to provide context for an LLM, enabling the generation of precise, source-grounded answers for complex queries on heatwave adaptation.}
		\item Supervised by Researcher \href{https://scholar.google.com/citations?user=OQDwnVUAAAAJ}{Yong Ge}.
	\end{itemize}

	\vspace{10pt}

	\textbf{Provincial Undergraduate Innovation Training Program} \hfill 2024.03 – 2025.06
	\begin{itemize}[noitemsep, topsep=0pt, partopsep=0pt, parsep=0pt, leftmargin=*]
		\item \textbf{Project:} Research on the Generation Cyclability of Cartesian Product Graphs and Related Problems.
		\item {\small The $k$-ary $n$-cube ($Q_n^k$) is a critical topology for large-scale computing systems powering modern AI and HPC workloads, where fault tolerance is paramount. Traditional fault models, which assume faults are independent and random, yield unrealistic resilience estimates because real-world failures are spatially correlated, manifesting as topological clusters. This paper introduces the Region-Based Fault (RBF) model, a new paradigm that addresses this gap by directly modeling this spatial correlation. Our primary contribution is a proof that for odd $k \geq 3$ and $n \geq 2$, the $Q_n^k$ remains Hamiltonian-connected—a property vital for deadlock-free routing and efficient task scheduling—under a sufficient set of RBF conditions. We present a constructive algorithm that finds a Hamiltonian path by leveraging an adaptive decomposition strategy. Experimental analysis demonstrates that our approach significantly enhances fault tolerance and remains robust far beyond its conservative theoretical guarantees. This work provides a practical, high-performance solution for maintaining connectivity in systems where fault clustering is prevalent.}
		\item \href{https://github.com/wyqmath/Hamiltonian_Path}{Project Repository}. Supervised by Prof. \href{https://orcid.org/0000-0003-1456-4539}{Eminjan Sabir}.
	\end{itemize}
	
	\vspace{12pt}
	
	\begin{center}
		\textbf{\large Professional Experience}
	\end{center}
	\textbf{Beijing Frontier Research Center for Biological Structure, Tsinghua University} \hfill Beijing, China
	
	\textit{Intern} \hfill 2025.07 – Present
	\begin{itemize}[noitemsep, topsep=0pt, partopsep=0pt, parsep=0pt, leftmargin=*]
		\item Developed the \href{https://www.fbs.frcbs.tsinghua.edu.cn/competition/2025Peptide/}{Peptide Design Competition Website}, the \href{https://www.frcbs.tsinghua.edu.cn/cpdb/}{Polypeptide Structure Database} and engaged in protein and polypeptide design research.
	\end{itemize}
	
	\vspace{10pt}
	
	\textbf{Institute of Software, CAS \& Huawei Mindspore Community} \hfill Remote
	
	\textit{Intern} \hfill 2024.09 – 2025.03
	\begin{itemize}[noitemsep, topsep=0pt, partopsep=0pt, parsep=0pt, leftmargin=*]
		\item Implemented a VGG19-based model for Pollock-style art generation, focusing on fractal and turbulent feature extraction and the creation of NFT-based digital labels. Featured on \href{https://mp.weixin.qq.com/s/_N5oLsa0etJTxEOKRTt_4w}{Huawei's official Wechat}.
	\end{itemize}
	
	\vspace{10pt}
	
	\vspace{12pt}
	
	\begin{center}
		\textbf{\large Selected Publications}
	\end{center}
	\begin{enumerate}[noitemsep, topsep=0pt, partopsep=0pt, parsep=0pt, leftmargin=*, label={[\arabic*]}]
		\item \textbf{Wang, Y.}, Cai, M., Dong, Y., et al. (2025). From Signal to Symphony: Exploring 2D Sequence Representations for Protein Function Prediction. \textit{Journal of Chemical Information and Modeling} (JCR Q1, CAS Q2 Top).
		\item \textbf{Wang, Y.}, Ma, Y., Chang, Y., et al. (2025). Diffusion Models at the Drug Discovery Frontier: A Review on Generating Small Molecules versus Therapeutic Peptides. \textit{arXiv preprint arXiv:2511.00209}.
		\item \textbf{Wang, Y.}, Zai, J., Liu, Z., et al. (2025). Resilient AI Infrastructure by Design: A Spatially-Aware Framework for Tolerating Clustered Failures. \textit{In 4th Annual AAAI Workshop on AI to Accelerate Science and Engineering (AI2ASE)}.
		\item \textbf{Wang, Y.}, Cai, M., Zhang, J., et al. (2025). Adaptive Decision-Making in Multi-Stage Production: A Framework for Cost Optimization under Sampling Uncertainty. \textit{Applied Operations and Analytics}.
		\item \textbf{Wang, Y.*}, Cai, M., \& Huang, T. Y. (2025). AI for disease prediction: Performance insights and key limitations. \textit{Journal of Clinical Neuroscience, 138}, 111360.
		\item \textbf{Wang, Y.*}, Huang, T. Y., Gao, Q., \& Zhang, J. (2025). HeDA: An Intelligent Agent System for Heatwave Risk Discovery through Automated Knowledge Graph Construction and Multi-layer Risk Propagation Analysis. \textit{arXiv preprint arXiv:2509.25112}.
		\item Wang, X., \textbf{Wang, Y.}, \& Huang, T. Y. (2025). Crypto-ncRNA: Non-coding RNA (ncRNA) Based Encryption Algorithm. \textit{ICLR 2025 Workshop}. (Co-first author).
		\item Wang, X., \textbf{Wang, Y.*}, Xu, L., et al. (2025). Octopus Inspired Optimization (OIO): A Hierarchical Framework for Navigating Protein Fitness Landscapes. \textit{In 2025 IEEE International Conference on Bioinformatics and Biomedicine (BIBM)}. IEEE. (CCF-B, Co-first author).
		\item Wang, X., \textbf{Wang, Y.*}, \& Pan, J. (2025). Digital Art Creation and Copyright Protection in Pollock Style Using GANs, Fractal Analysis, and NFT Generation. \textit{ICLR 2025 Workshop}. (Co-first author).
		\item \textbf{Wang, Y.*}, Zhang, J., \& Chang, Y. (2024, November). A probability prediction model for flood disasters based on Multi-layer Perceptron. In \textit{Journal of Physics: Conference Series} (Vol. 2905, No. 1, p. 012003). IOP Publishing.
	\end{enumerate}
	
	\vspace{12pt}
	
	\begin{center}
		\textbf{\large Awards \& Honors}
	\end{center}
	\begin{itemize}[noitemsep, topsep=0pt, partopsep=0pt, parsep=0pt, leftmargin=*]
		\item \textbf{Contemporary Undergraduate Mathematical Contest in Modeling (CUMCM)}: National Second Prize \hfill 2025
		\item \textbf{SynBio Challenges}: Silver Award $\times$ 2 \hfill 2025
		\item \textbf{2025 X-Fusion "Global Innovator Fusion Conference"}: Best Poster Award \hfill 2025
		\item \textbf{The Mathematical Contest in Modeling (MCM)}: Honorable Mention \hfill 2025
		\item \textbf{Alibaba Cloud Tianchi University Student Competition}: National Finals, 17th Place \hfill 2024
		\item \textbf{APMCM Asia-Pacific Mathematical Modeling Competition}: National Third Prize \hfill 2024
		\item \textbf{“Alpha Egg Cup” National Go Championship}: 15th Place \hfill 2024
		\item \textbf{Xinjiang Youth Amateur Go Competition}: 53rd Place \hfill 2024
		\item \textbf{Hunan Province Spring Cup Go Competition}: 7th Place \hfill 2024
		\item \textbf{National Youth Intellectual Sports Meeting Go Competition}: 9th Place \hfill 2024
		\item \textbf{Xinjiang University Vulnerability Reporting Honor} \hfill 2023
		\item \textbf{“Tianshan Fixed Network Cup” Cybersecurity Skills Competition}: 7th Place, Xinjiang Region \hfill 2023
	\end{itemize}

	\begin{center}
		\textbf{\large Academic Activities}
	\end{center}
	\textbf{Peer Reviewer:}
	\begin{multicols}{2}
		\begin{itemize}[noitemsep, topsep=0pt, partopsep=0pt, parsep=0pt, leftmargin=*]
			\item NeurIPS 2025 AI for Science Workshop
			\item NeurIPS 2025 MATH-AI Workshop
			\item ICLR 2025 Workshop on AI for Nucleic Acids
			\item ICLR 2025 Workshop on GenAI Watermarking
			\item ICML 2025 Workshop on AI for Math
			\item Mini-Reviews in Medicinal Chemistry
			\item Current Science
			\item F1000 Research
		\end{itemize}
	\end{multicols}
	
	\textbf{Academic Engagement:}
	\begin{itemize}[noitemsep, topsep=0pt, partopsep=0pt, parsep=0pt, leftmargin=*]
		\item Tsinghua University-Peking University Center for Life Sciences Summer Camp \hfill 2025.07
		\item \begin{tabular}[c]{@{}p{0.8\linewidth}@{}}Shenzhen Bay Laboratory / Shenzhen Medical Academy of Research and Translation Summer Research\end{tabular} \hfill \mbox{2025.07 -- 2025.09}
		\item Tsinghua University Tsien Excellence in Engineering Program -- Zero One Scholar \hfill 2024.06 -- 2027.06
		\item AI Winter School, Brown University Department of Physics \hfill 2025.01
		\item CAAI Artificial Intelligence and Technology Ethics Training Course \hfill 2024.09 -- 2024.12
		\item Fudan University Summer School of Mathematical Logic \hfill 2024.08
		\item Jinan University Guangdong Thousand Villages Survey Project \hfill 2024.08
		\item Wuhan University National Tianyuan Mathematics Center Discussion Class \hfill 2024.03 -- 2024.06
	\end{itemize}
	
	\vspace{12pt}

	\begin{center}
		\textbf{\large Website Development}
	\end{center}
	\begin{itemize}[noitemsep, topsep=0pt, partopsep=0pt, parsep=0pt, leftmargin=*]
		\item \textbf{Peptide Design Competition, Tsinghua University}: \href{https://www.fbs.frcbs.tsinghua.edu.cn/competition/2025Peptide/}{https://www.fbs.frcbs.tsinghua.edu.cn/competition/2025Peptide/}
		\item \textbf{Polypeptide Structure Database, Tsinghua University}: \href{https://www.frcbs.tsinghua.edu.cn/cpdb/}{https://www.frcbs.tsinghua.edu.cn/cpdb/}
		\item \textbf{Tong Wang (Tsinghua University) Research Group}: \href{https://tongwang.vercel.app/}{https://tongwang.vercel.app/}
	\end{itemize}

	\vspace{12pt}

	\begin{center}
		\textbf{\large Skills \& Interests}
	\end{center}
	\textbf{Technical Skills:} Python, C/C++, MATLAB, LaTeX, Linux, HTML/CSS/JavaScript
	
	\textbf{Research Interests:} Artificial Intelligence, Deep Learning, AI for Science, Bioinformatics, Computational Biology, Mathematical Modeling, Neuroscience
	
	\textbf{Languages:} Chinese (Native), English (Professional Working Proficiency)
	
\end{document}